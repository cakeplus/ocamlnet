\documentclass[12pt,a4paper]{article}

\usepackage{hyperref}

%%%%%%%%%%%%%%%%%%%%%%%%%%%%%%%%%%%%%%%%%%%%%%%%%%%%%%%%%%%%%%%%%%%%%%%%
\setlength{\parindent}{0pt}
\addtolength{\voffset}{-20ex}

\newcommand{\submiturl}{http://localhost/cgi/fdf\string_cgi.com}

\hypersetup{
  pdfpagemode=None,
  pdfauthor={Christophe Troestler},
  pdftitle={Form FDF update},
  pdfstartview=FitH
}
%%%%%%%%%%%%%%%%%%%%%%%%%%%%%%%%%%%%%%%%%%%%%%%%%%%%%%%%%%%%%%%%%%%%%%%%

\begin{document}

\begin{Form}[method=post, action={\submiturl}]
  \TextField[name=date,
  backgroundcolor={0.6 0.6 1.},bordercolor={0 0 0}]{Date~:}\\[2\jot]
  \TextField[name=time,
  backgroundcolor={0.6 0.6 1.},bordercolor={0 0 0}]{Time~:}\\[2\jot]
  % 
  As of now, \texttt{Netcgi} treats \texttt{application/vnd.fdf}
  data as a single \texttt{BODY} argument that is not parsed.
  Here is its value~:
  \\[1\jot]
  \TextField[name=args, width=0.8\linewidth, height=40ex, charsize=8pt,
  readonly, multiline,
  backgroundcolor={0.6 0.6 1.},bordercolor={0 0 0},
  value={Not submitted yet.}]{\texttt{BODY}~:}
  \\[2\jot]
  \Submit[bordercolor={0 0 1}]{\textit{Update}}\quad
  \Reset[bordercolor={0 0 1}]{\textit{Clear}}\\[2\jot]
\end{Form}
By pressing ``Update'' the above form data will be sent to
\begin{center}
  \texttt{\submiturl}  
\end{center}
which will send back a FDF data to update the above fields.

\end{document}
